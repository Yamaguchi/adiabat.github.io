\documentclass[11pt]{article}
\begin{document}

\section*{補足}

\subsection*{オラクルが公開するポイントRが1回限りである必要性について}

オラクルが公開するRは1回限りでなければいけません。まず、オラクルはランダムなシークレット \(k\) を用いて \(R\) を計算します。

\[R = kG\]

オラクルが同じシークレット \(k\) と \(R\) を用いて、署名を2回提出するとします(\(s_{1}, s_{2}\))。この時

\[s_{1} = k - h(m_{1}, R)v\]

\[s_{2} = k - h(m_{2}, R)v\]

となりますが、2つの式の両辺の差をとることにより、

\[s_{1} - s_{2} = (k - h(m_{1}, R)v) - (k - h(m_{2}, R)v) = -(h(m_{1}, R) - h(m_{2}, R))v \]

これは、オラクルの秘密鍵vが

\[v = -(s_{1} - s_{2})/(h(m_{1}, R) - h(m_{2}, R)) \]

で計算できてしまうことを意味します。

\end{document}